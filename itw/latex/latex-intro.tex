\documentclass[a4paper,10pt]{article}
\usepackage{graphicx}
\usepackage{latexsym,amsmath,amssymb}
%\usepackage{fontcolors}
\setlength{\textheight}{8.5in}
\setlength{\textwidth}{5.5in}
\setlength{\topmargin}{0.005in}
%opening
\title{To Test The Capabilities of the Human and The Tool}
\author{Anupama Potluri}

\begin{document}

\maketitle

\tableofcontents
\newpage

\begin{abstract}
This is to assess the capabilities of the tool, Microsoft's Word Processor, 
$MS-WORD$ and the student's comfort level with using it.
\end{abstract}

\section{A Few Aphorisms}

We will look at some of the font styles :

This is in \textit{italics}.
This is a \textsl{slanting} text.
This is a \textbf{bold face} text.
This is a \textsc{small caps} text.
\flushright This is flushed to the right.
\flushleft This is a text with flush to the left.
\begin{center}
 This is centered text.
\end{center}

\subsection{Large Font Section}
\large This is large text.
\Large This is even larger text.
\huge This is huge text.
\Huge This is even huger than the earlier one.

\normalsize If you want the text to come in a separate line, separate them by 
a line too.

\section{Enumeration with Latex}
In this section, we will see how to enumerate items. For example, I might 
want to enumerate the advantages and disadvantages of using Latex \cite{latex}. Then, I 
can do it as follows:

\begin{enumerate}
  \item {\bf Advantages:}
  \begin{itemize}
    \item It produces output of a publishable quality.
    \item It is very easy to do mathematical formulae.
    \item It is very easy to do cross-referencing which is almost a nightmare
	  in Microsoft Word and its clones.
  \end{itemize}
  \item {\bf Disadvantages:}
  \begin{itemize}
    \item Tables are not easy to do comparatively.
    \item Compilation and debugging are not always straightforward, 
	  especially for beginners.
  \end{itemize}

\end{enumerate}

More information may be obtained from the Latex Project \cite{latex-pro}. Citing
article in a journal \cite{art} or of a conference proceeding \cite{conf} may
also be done.

\section{Including Images}
\label{images}

We have a graph which represents a network topology in Fig. \ref{topo}.

\begin{figure}[ht]
\centering
%\includegraphics[scale=0.4]{topo.eps}
\caption{A topology}
\label{topo}
\end{figure}

If we take the topology in Fig. \ref{topo} from section \ref{images} and color 
them with different colors, we get the picture in Fig. \ref{color}.

\begin{figure}[ht]
\centering
%\includegraphics[scale=0.4]{topo5.eps}
\caption{A Color topology}
\label{color}
\end{figure}

\section{Mathematical Formulae}

All greek symbols\footnote{All mathematicians are fond of these} are very 
easy to write - such as $\alpha$, $\beta$, $\psi$ etc. Similarly, all 
mathematical relational symbols are also quite easy to do: $\neq$, 
$\leq$, $\geq$, $\simeq$, $\ll$, $\gg$ and so on. Set relationships are 
also easy: $\in$ to represent membership in a set, $\subset$ to represent 
subset,  Set Minus is $\setminus$ and so on.

A superscript $x^2$ and subscript $y_{z1}$ can be easily done.

A fraction is given as follows: $\frac{x+y}{2^{xy}}$

To get an equation on the next line, you can do the following:

\[
 \sum_x(x^2+1) = \sum_{x-1}x + f(y)
\]

To get an integral, you just say $\int_0^\infty x$.

You can get other special symbols using the package amssymb and amsmath - AMS 
standing for American Mathematical Society - which has defined these packages.

Thus you say $\mathbb{A, B, G}$ or $\mathcal{A, B, C}$ for different types of 
symbols.

Here is how you write a formula.

$\delta^\bullet: \mathbb{G}^\times \rightarrow \mathbb{G}^\bullet $ is such 
that $\delta^\bullet(X^\times) = \{x \in \mathbb{G}^\bullet \mid 
\exists e_{x,y} \in X^\times\}$ \vspace{2mm}

Let us now create an array:
\( \begin{array}{cccc}
 a & b     & c & d \\
 d     & e+f  & g^2 & h
\end{array} \)

We can put the whole of the above in square brackets as follows:
\(\left[ \begin{array}{cccc}
 a & b     & c & d \\
 d     & e+f  & g^2 & h
\end{array} \right] \)

Another example array with determinants etc.:
\[ \left( \begin{array}{c}
           \left| \begin{array}{cc}
                   x_{11} & x_{12} \\
                   x_{21} & x_{22} 
                  \end{array}
	    \right| \\
	    y \\
	    z
          \end{array}
    \right)
\]

You can have ellipsis in the text with lower alignment $\ldots$ or centered 
alignment as in $\cdots$.

A nice formula:
\[
 x = 
\left\{ 
\begin{array}{ll}
 y^2 & \mbox{if $y > 0$} \\
 y^{-2} & \mbox{otherwise}
\end{array}
\right.
\]

The same with an equation number:
\begin{equation}
 x = 
\left\{ 
\begin{array}{ll}
 y^2 & \mbox{if $y > 0$} \\
 y^{-2} & \mbox{otherwise}
\end{array}
\right.
\end{equation}

The {\bf stackrel} command stacks one symbol over another:
\( A \stackrel{a'}{\rightarrow} B \stackrel{b'}{\rightarrow} C \)  

\section{Creating Tables}

\begin{table*}[ht]
\centering
\caption{An Example multi-column Table with columns separated by different 
separators}
\begin{tabular}{|c | c || c | c || c | c|}
\hline 
\textbf{Column 1} & \textbf{Column 2} & \multicolumn{2}{c||} 
{\textbf{Column 3}} & \multicolumn{2}{c|} {\textbf{Column 4}} \tabularnewline
\cline{3-4} \cline{5-6} 
\multicolumn{1}{|c|}{} & \multicolumn{1}{c||}{} & \textbf{$\gamma$} & 
\textbf{Time (s)} & \textbf{$\gamma$} & \textbf{Time (s)} \tabularnewline
\hline
1 & 2 & 3 & 4 & 9 & 10 \\
5 & 6 & 7 & 8 & 11 & 12 \\
\hline
a & b & c & d & m & n\\
e & f & g & h & i & w \\
\hline
\end{tabular}  \vspace{1mm}
\label{table1}
\end{table*}

We show an example table in Table \ref{table1}.


\section{Advanced Features}
We can \fbox{create boxes} around text.

\makebox[15ex][s]{Censored text}\hspace{-15ex}\makebox[15ex][s]{X X X X X}


We can create theorem environment as follows:
\newtheorem{theorem}{Theorem}

\begin{theorem}
The sum of the squares of the sides of a right-angled triangle is equal to 
the square of the hypotenuse.
\end{theorem}

\newtheorem{axiom}{Axiom}

\fbox{
\parbox{\textwidth}{
  \begin{axiom} 
      Axiom scheme for Universal Instantiation.
  \end{axiom}
  }
}

\bibliographystyle{splncs}
\bibliography{latex-intro}

\end{document}
