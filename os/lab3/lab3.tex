\documentclass[a4paper,10pt]{article}
%\documentclass[a4paper,10pt]{scrartcl}
\usepackage[utf8]{inputenc}

\title{Operating Systems Lab-3 (Signals)}
\author{P. Anurag\\17MCME13}
\date{}

\pdfinfo{%
  /Title    (OS-Lab3)
  /Author   (Anurag Peddi)
  /Creator  ()
  /Producer ()
  /Subject  (Lab Submission)
  /Keywords (signals, kill)
}

\begin{document}
\maketitle
  \begin{enumerate}
    \item[Q1] I have sent a basic signal SIGINT(CTRL-C) by keeping it in a infinte loop, it handled it
    properly and the loop is still running.
    \item[Q2] We can handle every signal except two signals SIGKILL and SIGSTOP because these are kept
              by the operating systems for an emergency purpose; sending the kill -x pid (x is a signal)
              multiple times also doesn't effect the program, it's being handled.
    \item[Q3] If we send a signal to the program during it's sleep then OS immediately wakes up the process
              from the sleep and gives back the execution to the program.
    \item[Q4] I have opened firefox in background whose pid is 4797
              Then by typing pstree -s 4797 it listed all the parent \& child processes which 
              are associated to pid 4797\\
              The output is: \\
              $systemd----lightdm---lightdm---upstart---gnome-terminal----bash---firefox---Web-content ... $
              Bash: Shell on which I am working.
              gnome-terminal: Terminal emulation application which provides access to unix shell in GNOME environment.
              upstart: It is an event-based replacement for /sbin/init deamon.
              lightdm: It is a display manager.
              systemd: systemd is a system and service manager for linux operating system.
    \item[Q5] Except for the SIGCHILD all other signals are killing it.
              For SIGCHILD the default action is ignore so it's just ignoring the signal.
  \end{enumerate}

\end{document}
