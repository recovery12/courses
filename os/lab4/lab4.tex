\documentclass[a4paper,10pt]{article}
%\documentclass[a4paper,10pt]{scrartcl}
\usepackage[utf8]{inputenc}

\title{Operating Systems Lab-4 (Pipes)}
\author{P. Anurag\\17MCME13}
\date{}

\pdfinfo{%
  /Title    (OS-Lab4)
  /Author   (Anurag Peddi)
  /Creator  ()
  /Producer ()
  /Subject  (Lab Submission)
  /Keywords (Pipe, fork, read, write)
}

\begin{document}
\maketitle
  \begin{enumerate}
    \item[Q1] I have observed a major difference with wait() function
              if there is no wait() function the program takes only
              a part of the program and outputs the screen, if there is
              wait() then it outputs the complete txt file to the
              screen.
    \item[Q2] In this program, after giving a line for every newline
              character converts every lowercase character to uppercase
              stores it the buffer and outputs to stdout; from there 
              the child process takes each line, adds a newline to it and 
              then it outputs to the screen.
  \end{enumerate}

\end{document}
